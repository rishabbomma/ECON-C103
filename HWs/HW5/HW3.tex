\documentclass[11pt,oneside]{article}
\usepackage[hmargin=1in,vmargin=1in]{geometry}               % See geometry.pdf to learn the layout options. There are lots.
\geometry{letterpaper}                   % ... or a4paper or a5paper or ...
%\geometry{landscape}                % Activate for for rotated page geometry
%\usepackage[parfill]{parskip}    % Activate to begin paragraphs with an empty line rather than an indent
\usepackage{graphicx}
\usepackage{amssymb}
\usepackage{epstopdf}
\usepackage{url}
%\usepackage{verbatim}
\usepackage{comment}
\specialcomment{solution}{\textbf{Solution. }}{}
%\excludecomment{solution}    %uncomment to remove solutions.

%\usepackage{enumerate}

%Use the enumitem package instead of enumerate
\usepackage[shortlabels]{enumitem}
%\usepackage{enumitem}
%then it will support the same suntax as the enumerate package.
%The enumerate package does not provide any extra configurations other than the label.

%\setlist[enumerate]{topsep=0pt,itemsep=-1ex,partopsep=1ex,parsep=1ex}
\setlist[enumerate]{topsep=0pt,partopsep=0pt}

\DeclareGraphicsRule{.tif}{png}{.png}{`convert #1 `dirname #1`/`basename #1 .tif`.png}
\usepackage{amsmath,amsthm,amscd,amssymb}
\usepackage{latexsym}
\usepackage[colorlinks,citecolor=red,pagebackref,hypertexnames=false]{hyperref}

\numberwithin{equation}{section}

\theoremstyle{definition}
\newtheorem{exercise}{Exercise}
%\newtheorem{solution}{Solution}
\newtheorem*{defn}{Definition}


\def\calA{\mathcal{A}}
\def\calB{\mathcal{B}}
\def\calC{\mathcal{C}}
\def\calT{\mathcal{T}}
\def\OR{\overline{\mathbb{R}}}
\def\RR{\mathbb{R}}
\def\CC{\mathbb{C}}
\def\FF{\mathbb{F}}
\def\QQ{\mathbb{Q}}
\def\ZZ{\mathbb{Z}}
\def\NN{\mathbb{N}}
\def\KK{\mathbb{K}}
\def\PS{\mathfrak{P}}
\def\CS{\mathfrak{C}}

%\def\NN{\mathbb{Z}_{> 0}}
\def\Nzero{\mathbb{Z}_{\geq 0}}
\def\EE{\mathbb{E}}
\def\Pset{\mathbb{P}}
\def\supp{\mathrm{supp}}
\def\diam{\mathrm{diam}}
\def\sp{\mathrm{span}}
\def\ker{\mathrm{ker}}
\def\r{\hspace*{0.5mm} R \hspace*{0.5mm} }
\def\I{\hspace*{0.5mm} I \hspace*{0.5mm} }
\def\P{\hspace*{0.5mm} P \hspace*{0.5mm} }
%\def\sp{\mathrm{span}} %messes up align enviroment
\newcommand{\rbr}[1]{\left( {#1} \right)}
\newcommand{\sbr}[1]{\left[ {#1} \right]}
\newcommand{\cbr}[1]{\left\{ {#1} \right\}}
\newcommand{\abr}[1]{\left\langle {#1} \right\rangle}
\newcommand{\abs}[1]{\left| {#1} \right|}
\newcommand{\norm}[1]{\left\|#1\right\|}
\def\one{\mathbf{1}}
\DeclareMathOperator*{\esssup}{ess\,sup}
\newcommand*\wc{{}\cdot{}}
%\newcommand*\wc{ \, \cdot \,}
%wc for wildcard
\renewcommand{\Re}{\operatorname{Re}}
\renewcommand{\Im}{\operatorname{Im}}
\newcommand{\sgn}{\textup{sgn\,}}
\newcommand{\es}{\emptyset}
\newcommand{\IFF}{\Longleftrightarrow}
\newcommand{\openB}[2]{\text{B}_{#2}(#1)}
\newcommand{\RIGHT}{\Longrightarrow}
\newcommand{\LEFT}{\Longleftarrow}
\newcommand{\QED}{\blacksquare}
\newcommand{\intersect}{\bigcap}
\newcommand{\union}{\bigcup}
\newcommand{\closeB}[2]{\overline{B}_{#2}(#1)}
\newcommand{\close}[1]{\overline{#1}}
\newcommand{\inter}[1]{{#1}^\mathrm{o}}
\newcommand{\tend}{\longrightarrow}
\newcommand{\infnorm}[1]{\norm{#1}_{\infty}}
\newcommand{\pnorm}[2]{{\norm{#1}}_{#2}}
\newcommand{\lspace}[1]{{\ell}^{#1}}
\newcommand{\maps}[2]{L(#1,#2)}
\newcommand{\Gr}[1]{\Gamma{(#1)}}
\newcommand{\inner}[2]{\langle#1{,}#2\rangle}
\newcommand{\Span}[1]{\textnormal{span}(#1)}
\newcommand{\orgC}[1]{{#1}^{\perp}}
\newcommand{\tn}[1]{\textnormal{#1}}
\newcommand{\swf}{ \succsim}


\newcommand{\directsum}{\bigoplus}


\newcommand{\la}{\lambda}
\setlength{\parindent}{0pt}
\setlength{\parskip}{12pt}


%\title{\parbox{14cm}{\centering{  Interior points of circle and sphere packings}}}
\begin{document}

\textbf{HW 3 - Econ C103 - Fall 2024 - Rishab Bomma}

\begin{exercise}
Read the definition of the tournament SWF from the lecture notes. Which of the three properties: Rationality, IIA, and Unanimity, does the tournament SWF satisfy?

\end{exercise}

\begin{solution}

The tournament SWF is rational is since the relation $T(x,r) \leq T(y,r)$ is always complete and transitive since $T(\cdot,R)$ is always a natural number. The tournament SWF is also always unanimous since $|\{R_i | x R_i z|\}| \leq |\{R_i | y R_i z|\} 
 \leq |\{R_i | z R_i y|\}| \leq |\{R_i | z R_i y|\}|$ for when $x R_i y$ for all individuals and since $x$ is always preferred to $y$ pairwise and thus $T(x,R) > T(y,R)$. However, the tournament SWF doesn't satisfy IIA. Consider the two tables of different individual preferences that satisfy the hypothesis of IIA. The first one gives us a societal preference which is indifferent to all alternatives and the second one gives us a societal preference where $x$ is preferred to $y$.
 \begin{center}
\begin{tabular}{ c c c c c}
$R_1$ & $R_2$ & $R_3$  \\
$x$ & $y$ & $z$ \\
$y$ & $z$ & $x$   \\
$z$ & $x$ & $y$   

\end{tabular}
\end{center}

 \begin{center}
\begin{tabular}{ c c c c c}
$R_1$ & $R_2$ & $R_3$  \\
$z$ & $y$ & $z$ \\
$x$ & $z$ & $x$   \\
$y$ & $x$ & $y$   

\end{tabular}
\end{center}
 
 




\end{solution}

\begin{exercise}

For the case when $|X| = 2$ and $n \geq 2$, give an example of a non-dictatorial SWF that is complete, unanimous and is different from the majority rule. Briefly interpret the SWF you have constructed



\end{exercise}

\begin{solution}

For any profile, only consider the first $3$ individuals and then do pairwise majority rule. This will still be complete and unanimous, non-dictatorial, but not quite majority rule. 

\end{solution}

\begin{exercise}
Suppose that the society $N = \{1, . . . , n\}$ needs to determine a tax rate $t \in [0, 1]$. Each individual in the society has an income of $1$. If the tax rate is $t \in [0, 1]$, then all the collected tax money $tn$ is invested in a public project. If a total of $x$ dollars are invested in the public project and if individual $i \in N$ has $y_i$ to spend on her private consumption, then her utility is

\[u_i(x,y_i) = x^{\alpha_i} {y_i}^{1-\alpha_i} \] 

for some parameter $\alpha_i \in [0,1]$ that only individual $i$ knows
\begin{enumerate}

\item[(a)] Suppose individuals have no way to save, so each one of them spends all her income remaining from taxes to private consumption. Compute each individual’s utility when the tax rate is set to be $t \in [0, 1]$

\item[(b)] Let $n$ be odd. Is there a non-dictatorial and strategy-proof SCF $f$ that sets a tax rate $t = f (\alpha) \in [0, 1]$, as a function of the profile of utility parameters $\alpha = (\alpha_1, . . . , \alpha_n) \in {[0, 1]}^n$ and $\Im(f ) = [0, 1]$ (i.e. $f$ is onto)?


\end{enumerate}
\end{exercise}

\begin{solution}

\begin{enumerate}

\item[(a)] $u_i(x,y_i) = x^{\alpha_i} {(1-t)}^{1-\alpha_i}$
\item[(b)] If one were to just consider the median of all of the utility parameters then, by the non-manipulability of the 
Median Voter Scheme, this is a valid SCF.   \end{enumerate}



\end{solution}


\begin{exercise}

Let $n$ be odd. Show that when $X = [0, 1]$ and $R = (R_1, . . . , R_n)$ is a profile of single-peaked preferences, the pairwise majority outcome $\geq$ associated with $R$ is also single-peaked.


\end{exercise}

\begin{solution}

Since the interval is symmetric about the median let's only consider the case where $x < y \leq x^*$. The "best" case scenario in terms of the maximum of $|\{R_i | x R_i y\}|$ is the situation where $x$ is between $x_{m-2}$ and $x_{m-1}$ and $y$ is in-between $x_{m-1}$ and $x_m$ where $x_m = x^*$ is the median of the peaks of the preferences. In the "best" case for $x$, we still have that $y$ is preferred to $x$ via PM but that's the same outcome as the single peak preference with the median of the original peaks being that peak and thus the PM outcome of a profile of single-peaked preferences is single peaked $\QED$



\end{solution}

\begin{exercise}
Does there exist a profile of single-peaked preferences $R \in R^{*5}$ and alternatives $a, b, c, d \in [0, 1]$ such that:

\begin{center}
\begin{tabular}{ c c c c c}
$R_1$ & $R_2$ & $R_3$ & $R_4$ & $R_5$ \\
a & b & c & b & d \\
b & d & d & a & b \\
c & a & b & c & a \\
d & c & a & d &c
\end{tabular}
\end{center}

\end{exercise}
































\end{document}
